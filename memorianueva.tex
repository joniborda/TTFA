%----------------------------------------------------------------------------------------
% Preambulo y Configuración
%----------------------------------------------------------------------------------------

\documentclass[
    11pt,
    spanish,
    singlespacing,
    parskip,
    headsepline,
    bookmarks=true,
    unicode=true,
    pdftoolbar=true,
    pdfmenubar=true,
    pdffitwindow=false,
    colorlinks=true,
    linkcolor=blue,
    citecolor=blue,
    urlcolor=blue
]{MastersDoctoralThesis}

\usepackage[utf8]{inputenc} % Codificación de entrada UTF-8
\usepackage[T1]{fontenc}    % Codificación de salida para caracteres especiales
\usepackage{graphicx}       % Manejo de gráficos
\usepackage{eso-pic}        % Permite agregar fondos
\usepackage{hyperref}       % Manejo de hipervínculos y marcadores
\usepackage{tabularx}


% Redefinición de caracteres problemáticos en marcadores
\hypersetup{
    pdftitle={Título del Documento},
    pdfauthor={Autor del Documento},
    pdfkeywords={Especialización en Inteligencia Artificial},
    pdfstartview={FitH},
    unicode=true,
    colorlinks=true,
    linkcolor=blue,
    citecolor=blue,
    urlcolor=blue
}

\pdfstringdefDisableCommands{%
  \def\texttt#1{#1}%
  \def\textbf#1{#1}%
  \def\textit#1{#1}%
  \def\"{\"}%
  \def\~{~}%
  \def\'{'}%
  \def\^{}%
  \def\textunderscore{\_} % Manejo del subrayado en marcadores
}


% Definir comandos requeridos por la clase
\newcommand{\degreename}{Especialidad en Inteligencia Artificial} % Cambia según tu título
\newcommand{\univname}{Universidad de Buenos Aires} % Cambia según tu universidad
\newcommand{\keywordnames}{Palabras clave:}
%----------------------------------------------------------------------------------------
% Documento Principal
%----------------------------------------------------------------------------------------

\begin{document}

% Configuración de la portada
\posgrado{Carrera / Maestría}
\keywords{Especialización en Irnteligencia Artificial}

% Incluir la portada desde un archivo separado
\include{portada}

% Configuración del contenido preliminar
\frontmatter % Usar numeración romana para las páginas preliminares
\pagestyle{plain} % Estilo de encabezado simple

%----------------------------------------------------------------------------------------
% Resumen
%----------------------------------------------------------------------------------------

\begin{abstract}
\addchaptertocentry{\abstractname} % Agregar resumen al índice
En la presente memoria se describe el desarrollo de un modelo de inteligencia artificial para el pronóstico de ventas en la tienda en línea de la empresa Latech, dedicada a la producción y comercialización de barritas alimenticias. El trabajo tiene como finalidad mejorar la planificación de la producción y la gestión del inventario mediante predicciones precisas basadas en datos históricos de ventas y publicidad.

Para su realización se aplicaron conocimientos de análisis de datos, modelado de series temporales, aprendizaje automático, integración de APIs y desarrollo de software, con el propósito de generar una herramienta que sirva de apoyo a la toma de decisiones estratégicas en la empresa.


\end{abstract}

%----------------------------------------------------------------------------------------
% Agradecimientos
%----------------------------------------------------------------------------------------
%---
%%\begin{acknowledgements}
%\vspace{1.5cm}
%Esta sección es para agradecimientos personales y es totalmente \textbf{OPCIONAL}.
%\end{acknowledgements}
%
%----------------------------------------------------------------------------------------
% Índice
%----------------------------------------------------------------------------------------

\tableofcontents
\listoffigures
\listoftables

%----------------------------------------------------------------------------------------
% Capítulos
%----------------------------------------------------------------------------------------

\mainmatter % Iniciar numeración numérica para el contenido principal
\pagestyle{thesis} % Estilo de encabezado de tesis

% Incluir capítulos desde archivos separados
% Chapter 1

\chapter{Introducción general} % Main chapter title

\label{Chapter1} % For referencing the chapter elsewhere, use \ref{Chapter1} 
\label{IntroGeneral}

En este capítulo se presenta el contexto general del trabajo y la problemática que motivó su desarrollo. Se exponen las razones y necesidades que impulsaron su realización, junto con una descripción de las soluciones existentes y los enfoques actuales relacionados con la temática abordada. Asimismo, se detallan los propósitos principales del trabajo y se delimitan los alcances y límites de su implementación.

%----------------------------------------------------------------------------------------

% Define some commands to keep the formatting separated from the content 
\newcommand{\keyword}[1]{\textbf{#1}}
\newcommand{\tabhead}[1]{\textbf{#1}}
\newcommand{\code}[1]{\texttt{#1}}
\newcommand{\file}[1]{\texttt{\bfseries#1}}
\newcommand{\option}[1]{\texttt{\itshape#1}}
\newcommand{\grados}{$^{\circ}$}

%----------------------------------------------------------------------------------------
\section{Contexto del Trabajo}

El presente trabajo forma parte de la Carrera de Especialización en Inteligencia Artificial y tiene como propósito el desarrollo de un modelo de inteligencia artificial capaz de pronosticar las ventas de una tienda en línea perteneciente a la empresa Latech.
La empresa fabrica y comercializa barritas alimenticias en distintos sabores, lo que requiere una planificación de producción precisa para evitar tanto faltantes como excedentes de stock. En este contexto, contar con pronósticos confiables de ventas constituye un factor estratégico clave para la toma de decisiones operativas y comerciales.

El trabajo se desarrolló con financiamiento de la empresa Latech y tuvo un presupuesto estimado de 684 horas y un costo aproximado de \$12.825.000, con fecha de inicio el 24 de junio de 2025 y presentación pública prevista para abril de 2026.

Para la elaboración del modelo se dispuso del acceso a la API de Shopify, plataforma en la nube que aloja la tienda del cliente, y a la API de Triple Whale, una herramienta especializada en la gestión y análisis de inversión publicitaria. Estas fuentes de datos permitieron obtener información histórica de aproximadamente dos años sobre ventas e inversión en publicidad digital.


\begin{figure}[htbp]
	\centering
	\includegraphics[width=1\textwidth]{./Figures/diagBloques.png}
	\caption{Diagrama en bloques.}
	\label{fig:texmaker}
\end{figure}


El sistema desarrollado se integró dentro del entorno interno del cliente, denominado Inventory Tracker, a fin de generar pronósticos de ventas a partir de los datos recopilados. En la Figura 1 se presenta el diagrama en bloques del sistema, donde se observa el flujo de información desde las APIs externas hasta la generación de los resultados.


\section{Motivación}

El desarrollo de este trabajo surge de la necesidad de la empresa Latech de disponer de una herramienta que le permita anticipar el comportamiento de las ventas de su tienda en línea y, de esta manera, mejorar la planificación de la producción y la gestión del stock.

La falta de un sistema de pronóstico confiable generaba dificultades en la toma de decisiones estratégicas y operativas, esto afectaba tanto la disponibilidad de productos como la eficiencia de las campañas de marketing. Por ello, se consideró necesario diseñar una solución basada en técnicas de inteligencia artificial que proporcionara predicciones con un alto grado de precisión y permitiera analizar la evolución de las ventas en el tiempo.

Asimismo, se buscó analizar el comportamiento de los usuarios. Se observó la cantidad de clientes recurrentes y los canales de venta utilizados. Los datos indicaron que las compras se realizaban a través de diferentes plataformas: Skio, un servicio que gestiona pagos y ventas adicionales en la tienda; TikTok, Facebook, Instagram y la propia tienda alojada en Shopify. Esta información resultó fundamental para comprender los patrones de compra y optimizar las predicciones del modelo.

Además, el uso de fuentes de datos integradas, como Shopify y Triple Whale, posibilitó aprovechar información histórica de ventas, inversión publicitaria y comportamiento de los usuarios, lo que resultó esencial para la construcción de un modelo predictivo robusto y adaptable a las necesidades del cliente.

En este sentido, el trabajo buscó no solo resolver una problemática específica de la empresa, sino también demostrar el potencial del aprendizaje automático aplicado al comercio electrónico como herramienta de apoyo a la toma de decisiones empresariales.



\section{Estado del arte}

En la actualidad, el pronóstico de ventas mediante técnicas de inteligencia artificial constituye un área de creciente interés en el ámbito del comercio electrónico. Los avances en el análisis de datos y en el aprendizaje automático han permitido el desarrollo de modelos capaces de anticipar la demanda con altos niveles de precisión. Esto contribuye a una mejor planificación de la producción, la gestión de inventarios y la toma de decisiones estratégicas.

Entre los enfoques más utilizados se encuentran los modelos de series temporales clásicos, como ARIMA, SARIMA y \textit{Exponential Smoothing}, que permiten capturar patrones estacionales y tendencias a lo largo del tiempo. Sin embargo, estos métodos suelen presentar limitaciones cuando los datos incluyen múltiples factores externos, como campañas publicitarias, canales de venta o comportamiento del usuario.

En respuesta a esas limitaciones, en los últimos años se ha incrementado el uso de técnicas de aprendizaje profundo (\textit{Deep Learning}) y aprendizaje automático (\textit{Machine Learning}), que ofrecen una mayor capacidad para modelar relaciones no lineales y para integrar múltiples fuentes de información. Modelos como Redes Neuronales Recurrentes (RNN), LSTM (\textit{Long Short-Term Memory}) y \textit{Random Forest Regressor} se han aplicado con éxito en la predicción de ventas en entornos de comercio electrónico, debido a su habilidad para capturar dependencias temporales y correlaciones entre variables.

Asimismo, el uso de plataformas integradas de datos, tales como Shopify y Triple Whale, ha impulsado el desarrollo de soluciones personalizadas que combinan información transaccional, métricas de marketing y comportamiento de los usuarios. Estas herramientas permiten no solo mejorar la precisión de los modelos de pronóstico, sino también ofrecer una visión integral del desempeño comercial y de la efectividad de las campañas publicitarias.

Entre las soluciones comerciales más reconocidas se destacan Amazon Forecast, un servicio basado en redes neuronales desarrollado por Amazon Web Services que automatiza la creación de modelos de predicción de demanda, y Prophet, una herramienta de código abierto creada por Meta (Facebook) que utiliza un enfoque aditivo para modelar tendencias y estacionalidades de manera flexible. Ambas herramientas representan referencias relevantes en el campo del pronóstico de series temporales con datos de negocios.

Diversos estudios recientes también han explorado la aplicación de modelos híbridos que combinan técnicas tradicionales y redes neuronales para mejorar la precisión de los resultados. Estos enfoques se han implementado especialmente en sectores minoristas, donde la influencia de factores externos —como la publicidad digital, las promociones y la estacionalidad— tiene un peso significativo en la dinámica de ventas.

En este marco, el trabajo desarrollado se apoyó en los enfoques actuales de predicción de demanda mediante aprendizaje automático, incorporando datos históricos de ventas, inversión publicitaria y comportamiento del usuario. El objetivo fue construir un modelo que reflejara las tendencias reales del negocio y sirviera como herramienta de apoyo para la toma de decisiones estratégicas en la empresa Latech.


\section{Objetivo y Alcance}

En esta sección se mencionan los propósitos principales del trabajo y los límites de su implementación.

\subsection{Objetivo del trabajo}

El propósito de este trabajo es desarrollar un modelo de inteligencia artificial que permita predecir con precisión las ventas del producto alimenticio de la empresa Latech, a partir de datos de ventas y de inversión publicitaria. Esto permitirá a la empresa: anticipar sus necesidades de
producción, optimizar la planificación del inventario y tomar decisiones estratégicas basadas
en datos. Asimismo, podrá reducir el riesgo de faltantes o excesos de inventario, y mejorar la
eficiencia operativa y comercial.

\subsection{Alcance del trabajo}

A continuación, se enuncia los items que incluye el trabajo:

\begin{itemize}
    \item Relevamiento y análisis de los datos históricos de ventas de la tienda en línea obtenidos mediante la API de Shopify.
    
    \item Relevamiento y análisis de los datos históricos de inversión publicitaria provenientes de la
API de Triple Whale.

    \item Limpieza, transformación y consolidación de datos provenientes de ambas plataformas.
    
     \item Análisis exploratorio de datos (EDA) para identificar patrones, tendencias y relaciones
entre las variables.
     
     \item Desarrollo y entrenamiento de modelos de inteligencia artificial orientados a la predicción
de ventas.
     
     \item Evaluación comparativa de distintos modelos para seleccionar el que ofrezca la mejor
precisión y capacidad predictiva.
     
     \item Implementación de un módulo funcional integrado al sistema Inventory Tracker de la
empresa, que permita realizar predicciones de ventas para períodos futuros.

     
     \item Documentación técnica del proceso, del modelo elegido y de su uso.
     
     \item Entrega de reportes que expliquen los hallazgos y recomendaciones derivadas del análisis.
     
\end{itemize}

A modo de limitaciones en la implementación se enuncian loa items que no incluye el trabajo:

\begin{itemize}
    \item Implementaciones en tiempo real del modelo, es decir, sistemas que actualicen las
predicciones de forma instantánea ante cada nuevo dato recibido. Sí se contempla la
posibilidad de programar ejecuciones periódicas del modelo (por ejemplo, una vez al día)
para actualizar las predicciones de manera regular.
	\item Garantía de precisión absoluta en las predicciones, ya que la precisión depende de la
calidad y estabilidad de los datos futuros y de factores externos no controlables.
	\item Acciones o recomendaciones específicas sobre estrategias de marketing más allá de lo
inferido de los análisis de datos.

	\item Optimización de procesos internos de producción o logística, salvo en lo que respecta a la estimación de demanda.

\end{itemize}

\subsection{Condiciones del trabajo}

El desarrollo del presente trabajo se realizó considerando las siguientes condiciones:

\begin{itemize}
    \item La empresa Latech proporcionó acceso completo y continuo a las APIs externas de Shopify y Triple Whale, lo que permitió obtener datos históricos y actualizados.

    \item Se contó con acceso al repositorio de código y a la infraestructura necesaria para integrar el módulo de predicción en el sistema existente Inventory Tracker.

    \item La calidad, integridad y consistencia de los datos obtenidos desde las plataformas externas fue suficiente para entrenar y validar los modelos de predicción.

    \item Durante el período de desarrollo no se produjeron cambios significativos en las políticas de acceso ni en la estructura de datos de las APIs utilizadas.

    \item No existieron restricciones legales o contractuales que impidieran el uso de los datos necesarios para el trabajo.

\end{itemize}

\chapter{Introducción específica} % Main chapter title

\label{Chapter2}

En este capítulo se presentan los requerimientos más importantes, los modelos de inteligencia artificial utilizados y las herramientas usadas para el tratamiento de datos.

\section{Requerimientos}

En esta sección se presentan algunos requerimientos establecidos para el trabajo. En cada requerimiento se detalla su objetivo y alcance, con el fin de precisar las condiciones necesarias para el desarrollo y la correcta implementación del modelo propuesto.


	\begin{itemize}
		\item Importación de datos de ventas desde Shopify: el sistema debe permitir la obtención de datos históricos y actualizados de ventas desde la plataforma Shopify, con el objetivo de utilizarlos en el proceso de entrenamiento del modelo de predicción de demanda.
		\item Importación de datos de inversión publicitaria desde Triple Whale: el sistema debe permitir la incorporación de datos de inversión publicitaria provenientes de la plataforma Triple Whale, con el propósito de integrar esta variable en el análisis y modelado de ventas.
		\item Consolidación de datos provenientes de Shopify y Triple Whale: el sistema debe unificar la información de ventas y de inversión publicitaria en un único conjunto de datos, con el fin de preparar una base consistente para el entrenamiento del modelo de predicción.
		\item Desarrollo y entrenamiento del modelo predictivo: el sistema debe implementar y entrenar un modelo de inteligencia artificial capaz de generar predicciones de ventas precisas sobre los datos históricos consolidados.
		\item Visualización de predicciones de ventas en Inventory Tracker: el sistema debe presentar las predicciones de ventas generadas por el modelo dentro de la plataforma Inventory Tracker, integradas con los datos históricos para facilitar el análisis y la planificación de producción.
		\item Descarga de predicciones de ventas en formato CSV: el sistema debe permitir la exportación de las predicciones de ventas a archivos CSV para su análisis externo o integración con otras herramientas de planificación.
	\end{itemize}




\section{Modelos de inteligencia artificial utilizados}
\label{sec:modelo-ia}

Para abordar el trabajo se evaluaron y aplicaron diversos modelos de inteligencia artificial, dado que el comportamiento de la demanda no responde a un único patrón fijo ni puede ser capturado adecuadamente por un solo tipo de modelo. También, se consideró seleccionar aquel que ofreciera el mejor equilibrio entre precisión, interpretabilidad y capacidad de generalización.

A continuación, se describen los tipos de modelos aplicados:

	\begin{itemize}
		\item Random Forest Regressor \citep{randomforestregressor}: se utilizó por su robustez frente a sobreajuste y su capacidad para capturar relaciones no lineales entre variables. Este modelo resulta especialmente útil cuando se integran múltiples fuentes de datos, como ventas históricas e inversión publicitaria.

		\item Gradient Boosting Machines \cite{friedman2001gbm} (XGBoost, LightGBM): se evaluaron por su alto rendimiento en competencias de ciencia de datos y su eficiencia computacional. Estos modelos permiten optimizar la función de pérdida de manera iterativa y mejorar la precisión en la predicción de series temporales con características externas.

	\item SARIMA (\textit{Seasonal Autoregressive Integrated Moving Average}): se aplicó para modelar la estacionalidad y tendencia presentes en los datos de ventas. SARIMA es especialmente efectivo cuando el comportamiento temporal es predominante y estable a lo largo del tiempo.

	\item \textit{Exponential Smoothing} (ETS) \citep{hyndman2008forecasting}: se consideró como alternativa para capturar patrones estacionales y de tendencia con un enfoque más intuitivo y menos paramétrico que SARIMA.

	\item \textit{Long Short-Term Memory} (LSTM): se implementaron redes LSTM debido a su habilidad para retener información a largo plazo y modelar dependencias temporales complejas. Este tipo de red es adecuado para series temporales con patrones no lineales y múltiples variables exógenas, como la inversión publicitaria por plataforma.

	\item GRU (\textit{Gated Recurrent Unit}) \citep{cho2014gru}: se evaluó como una variante más simple y computacionalmente eficiente de las LSTM, útil cuando los recursos de entrenamiento son limitados. Además, mantienen la capacidad esencial de capturar dependencias temporales a largo plazo y manejar el problema del gradiente \textit{vanishing}, que en ocasiones demuestra un desempeño comparable al de las LSTM.

	\item \textit{Temporal Fusion Transformer} (TFT): Se implementó este modelo de vanguardia por su capacidad de atención interpretable, que permite identificar qué variables exógenas y qué períodos históricos influyen en cada predicción. El TFT resulta particularmente adecuado para el contexto de Latech, donde es crucial entender el impacto de las inversiones publicitarias en diferentes horizontes temporales.

	\item Enfoque híbrido: se exploró la combinación de modelos clásicos (como SARIMA) con redes neuronales (como LSTM o TFT), con el fin de aprovechar la capacidad de los primeros para modelar componentes estacionales y de tendencia, y la flexibilidad de las segundas para capturar relaciones no lineales y efectos de variables externas.

	\item Prophet: se incluyó el modelo Prophet, desarrollado por Meta \citep{meta}, por su facilidad de uso y capacidad para manejar estacionalidades múltiples, días festivos y cambios en la tendencia. Este modelo resultó de interés para validar la estructura temporal de los datos antes de aplicar modelos más complejos.
	\end{itemize}

Cada uno de estos modelos fue entrenado y validado utilizando el conjunto de datos consolidado de ventas e inversión publicitaria, y se compararon mediante métricas como MAE (error absoluto medio), RMSE (raíz del error cuadrático medio) y MAPE (error porcentual absoluto medio), con el fin de seleccionar el que mejor se adaptara a las necesidades de pronóstico de Latech.


\section{Tratamiento de datos (herramientas)}

En este a sección se describen los procesos y herramientas utilizadas para la limpieza, transformación y análisis de los datos, así como los recursos externos desarrollados por terceros que resultaron fundamentales para la obtención y procesamiento de la información.




\subsection{Bibliotecas para procesamiento y análisis de datos}
\begin{itemize}
	\item Pandas \citep{pandas}: permite manipular y transformar datos tabulares, incluidas operaciones de filtrado, agrupamiento y combinación de datasets.

	\item NumPy \citep{numpy}: ofrece operaciones numéricas y manejo eficiente de arreglos multidimensionales.

	\item Scikit-learn \citep{sklearn}: empleada para el preprocesamiento de datos (escalado, codificación de variables categóricas) y la implementación de modelos clásicos de \textit{machine learning}.
	
	\item Pytorch \citep{pytorch}: framework que permite definir arquitecturas personalizadas, calcular gradientes automáticamente y realizar entrenamientos acelerados mediante GPU. Su integración con bibliotecas como NumPy, Pandas y Matplotlib facilita incorporarlo a flujos de trabajo existentes y explorar modelos predictivos más sofisticados que los basados únicamente en series temporales clásicas.

\end{itemize}


\subsection{Bibliotecas para visualización}

\begin{itemize}
		\item Matplotlib \citep{matplotlib} y Seaborn \citep{seaborn}: permiten generar gráficos estáticos que facilitan la identificación de patrones, tendencias y valores atípicos.
		
		

\item Plotly \citep{plotly}: utilizado para la creación de visualizaciones interactivas dentro de los notebooks.
\end{itemize}


\subsection{Plataformas y APIs externas}
\begin{itemize}
	\item Shopify API \citep{shopify}: proporciona acceso automatizado a datos históricos de ventas, productos y transacciones.
	
	\item Triple Whale API \citep{triplewhale}: ofrece datos de inversión publicitaria desglosados por plataforma y campaña.
	
	\item PostgreSQL \citep{postgresql_docs}: funciona como sistema de gestión de bases de datos destinado al almacenamiento estructurado de los datos consolidados.

	\item Cron \citep{cron_docs}: permite ejecutar tareas programadas a horas, fechas o intervalos fijos periódicos para automatización de \textit{pipelines}.
	
	\item AWS S3 \citep{aws_s3}: ofrece un servicio de almacenamiento de objetos en la nube de AWS que con alta escalabilidad, disponibilidad, durabilidad y seguridad necesarios para guardar los modelos y datos intermedios.
	
	\item Git \citep{git_pro}: sistema de control de versiones distribuido que mantiene un registro histórico de cambios en archivos de código, configuraciones y documentación, esencial para el desarrollo colaborativo y la reproducibilidad de experimentos.
\end{itemize}


\subsection{Exposición de servicios}
Para exponer endpoints que permiten consumir resultados, ejecutar procesos o servir modelos se utiliza FastAPI. Este framework destaca por su alto rendimiento y por su arquitectura asíncrona basada en ASGI \citep{asgi}.

Las características más relevantes para este proyecto son:
\begin{itemize}
	\item Alto rendimiento: maneja múltiples solicitudes de manera eficiente gracias a su soporte nativo para operaciones asíncronas.
	\item Validación y documentación automática: incorpora Pydantic  \citep{pydantic} para validar estructuras de entrada y genera documentación OpenAPI/Swagger \citep{swagger} sin requerir configuraciones adicionales.
	\item Flexibilidad: permite definir endpoints destinados al consumo de modelos, consultas filtradas, refresco de datos o ejecución de procesos administrativos.
\end{itemize}


\subsection{Contenerización}
\begin{itemize}
	\item Docker \citep{docker}: permite definir entornos de ejecución consistentes para los servicios del proyecto. Sus contenedores aseguran reproducibilidad, aislamiento y portabilidad entre distintos entornos.
	
	\item Docker compose \citep{docker_compose}: herramienta que simplifica la orquestación de múltiples contenedores Docker mediante archivos de configuración en formato YAML. Permite la definición y ejecución de aplicaciones multi-servicio de manera coordinada, lo que facilita la puesta en marcha de entornos complejos que incluyen la API de FastAPI, la base de datos PostgreSQL y servicios auxiliares de forma integrada. Docker Compose maneja automáticamente las dependencias entre servicios, las redes virtuales y los volúmenes de persistencia.
\end{itemize}

\subsection{Orquestación y pipelines}
\begin{itemize}
	\item Metaflow \citep{metaflow}: se utiliza como herramienta principal para estructurar y ejecutar pipelines de datos y experimentos, con registro automático de artefactos y trazabilidad.

	\item Apache Airflow \citep{airflow}: framework de orquestación que define flujos de trabajo como DAGs (\textit{Directed Acyclic Graphs}) \citep{airflow_dags} y permite programar ejecuciones periódicas, manejar dependencias entre tareas y monitorizar el estado de los procesos de datos.
	
\end{itemize}

\chapter{Diseño e implementación} % Main chapter title

\label{Chapter3} % Change X to a consecutive number; for referencing this chapter elsewhere, use \ref{ChapterX}

\definecolor{mygreen}{rgb}{0,0.6,0}
\definecolor{mygray}{rgb}{0.5,0.5,0.5}
\definecolor{mymauve}{rgb}{0.58,0,0.82}

%%%%%%%%%%%%%%%%%%%%%%%%%%%%%%%%%%%%%%%%%%%%%%%%%%%%%%%%%%%%%%%%%%%%%%%%%%%%%
% parámetros para configurar el formato del código en los entornos lstlisting
%%%%%%%%%%%%%%%%%%%%%%%%%%%%%%%%%%%%%%%%%%%%%%%%%%%%%%%%%%%%%%%%%%%%%%%%%%%%%
\lstset{ %
  backgroundcolor=\color{white},   % choose the background color; you must add \usepackage{color} or \usepackage{xcolor}
  basicstyle=\footnotesize,        % the size of the fonts that are used for the code
  breakatwhitespace=false,         % sets if automatic breaks should only happen at whitespace
  breaklines=true,                 % sets automatic line breaking
  captionpos=b,                    % sets the caption-position to bottom
  commentstyle=\color{mygreen},    % comment style
  deletekeywords={...},            % if you want to delete keywords from the given language
  %escapeinside={\%*}{*)},          % if you want to add LaTeX within your code
  %extendedchars=true,              % lets you use non-ASCII characters; for 8-bits encodings only, does not work with UTF-8
  %frame=single,	                % adds a frame around the code
  keepspaces=true,                 % keeps spaces in text, useful for keeping indentation of code (possibly needs columns=flexible)
  keywordstyle=\color{blue},       % keyword style
  language=[ANSI]C,                % the language of the code
  %otherkeywords={*,...},           % if you want to add more keywords to the set
  numbers=left,                    % where to put the line-numbers; possible values are (none, left, right)
  numbersep=5pt,                   % how far the line-numbers are from the code
  numberstyle=\tiny\color{mygray}, % the style that is used for the line-numbers
  rulecolor=\color{black},         % if not set, the frame-color may be changed on line-breaks within not-black text (e.g. comments (green here))
  showspaces=false,                % show spaces everywhere adding particular underscores; it overrides 'showstringspaces'
  showstringspaces=false,          % underline spaces within strings only
  showtabs=false,                  % show tabs within strings adding particular underscores
  stepnumber=1,                    % the step between two line-numbers. If it's 1, each line will be numbered
  stringstyle=\color{mymauve},     % string literal style
  tabsize=2,	                   % sets default tabsize to 2 spaces
  title=\lstname,                  % show the filename of files included with \lstinputlisting; also try caption instead of title
  morecomment=[s]{/*}{*/}
}

En este capítulo se describen los criterios utilizados para la construcción del desarrollo y la arquitectura definida para la solución. Se presenta la estructura general del sistema, las decisiones de diseño adoptadas y los componentes que permiten integrar fuentes de datos externas, procesarlas y generar pronósticos confiables de ventas.

\section{Arquitectura del sistema}

La arquitectura diseñada para el sistema de predicción de ventas se basa en un flujo de procesamiento secuencial y automatizado que conecta las fuentes de datos externas con el entorno interno del cliente. El diseño se orientó por tres criterios principales:

\begin{itemize}


\item Centralización y coherencia de datos. Se priorizó la unificación de múltiples fuentes (ventas, comportamiento de usuarios, inversión en publicidad).

\item Escalabilidad operativa que permita que el sistema integre nuevos canales, nuevas métricas o nuevos modelos sin alterar la estructura existente.

\item Procesos reproducibles y auditables para asegurar que cada etapa sea trazable y que el sistema pueda ser mantenido o ampliado por equipos técnicos futuros.

\end{itemize}

El sistema se organiza en una arquitectura por capas que separa claramente las responsabilidades, desde la obtención de los datos hasta la entrega del pronóstico al usuario final. Estas capas son:

\begin{enumerate}

\item Capa de recolección de datos (Data Ingestion Layer).

\item Capa de procesamiento y transformación (ETL).

\item Capa de almacenamiento estructurado (Data Storage Layer).

\item Capa de modelado predictivo (Modeling Layer).

\item Capa de integración operativa mediante microservicios.

\item Capa de visualización y consumo dentro de Inventory Tracker.

\end{enumerate}

Cada una de estas capas se comunica de manera acotada mediante APIs internas o a través de la base de datos para garantizar un bajo acoplamiento.


\subsection{Capa de recolección de datos}

Esta capa se encarga de conectarse periódicamente a APIs externas para obtener la información necesaria para entrenar y actualizar los modelos. Las dos fuentes externas son:
\begin{itemize}
	\item Shopify: funciona como el pilar central de la información comercial del sistema. A través de su API es posible acceder a los históricos de ventas, que incluye detalles como estado, valor total y descuentos. Además, cada orden incluye etiquetas que permiten distinguir si la compra corresponde a un cliente nuevo o recurrente, lo que posibilita analizar el comportamiento de los usuarios a lo largo del tiempo. Esta distinción es un insumo clave para caracterizar patrones de fidelidad, frecuencia de compra y recurrencia. A partir de la combinación de órdenes, etiquetas de comportamiento y series temporales de ventas, se logra reconstruir el ciclo completo de los clientes y diferenciar entre la demanda estable generada por suscriptores activos y la demanda variable asociada a compras espontáneas o impulsadas por campañas de marketing.

	\item Triple Whale: una plataforma que centraliza la información de inversión publicitaria proveniente de TikTok Ads \citep{tiktokAds}, Instagram Ads \citep{InstagramAds}, Facebook Ads \citep{FacebookAds} y Google Ads \citep{GoogleAds}. En este caso, su API no provee datos desagregados de campañas individuales, sino únicamente el monto diario total invertido en publicidad considerando todas las campañas activas. A pesar de esta limitación, esta información resulta suficiente para analizar la relación entre los niveles diarios de inversión publicitaria y las variaciones observadas en las ventas. De esta forma, el gasto diario consolidado funciona como un indicador de la presión publicitaria ejercida en cada jornada, lo que permite estudiar su impacto en el comportamiento de compra y en la demanda general del sistema.
\end{itemize}

Ambas integraciones están preparadas para ejecutarse en un proceso automatizado que consulta las APIs diariamente y almacena la información en una base de datos PostgreSQL \citep{postgresql_docs} interna. Este proceso opera bajo un módulo ETL \citep{microsoft_etl} donde se llevan a cabo tareas de validación, normalización y estandarización para garantizar la coherencia entre fuentes heterogéneas.

La decisión de almacenar localmente los datos en PostgreSQL responde a tres objetivos principales:

\begin{itemize}
    \item Autonomía del sistema: aunque se dispone de acceso a las APIs de Shopify y Triple Whale, estas pueden presentar interrupciones temporales, límites de tasa de consulta (\textit{rate limits}) o cambios no anunciados en su estructura. Al mantener una copia local de los datos, el sistema de pronóstico continúa funcionando incluso durante caídas de las APIs externas, garantizando disponibilidad continua.

    \item Rendimiento y velocidad: consultar datos históricos desde una base de datos local es significativamente más rápido que realizar múltiples solicitudes HTTP a APIs externas, especialmente cuando se procesan grandes volúmenes de datos durante el entrenamiento de modelos o la generación de informes.

    \item Desacoplamiento y control de datos: al poseer una copia estructurada de los datos, se evita la dependencia directa de la disponibilidad y formato de las APIs externas. Esto permite realizar transformaciones, enriquecimientos y agregaciones previas que optimizan los procesos posteriores de modelado y análisis.
\end{itemize}

De esta manera, PostgreSQL actúa como una capa de abstracción y resiliencia, que permite que el sistema opere de manera eficiente, predecible y con menor latencia, sin comprometer su funcionalidad ante eventuales fallos externos.

\subsection{Capa de procesamiento y transformación}

Una vez obtenidos los datos desde las APIs externas, se realiza un proceso de transformación que normaliza y estructura la información. Las principales tareas incluyen:

\begin{itemize}

\item Limpieza de campos inconsistentes o incompletos.

\item Conversión de formatos de fechas, monedas y valores numéricos.

\item Enriquecimiento de datos combinando ventas, comportamiento de usuarios y gasto publicitario.

\item Integración de información transaccional con métricas de suscripción.

\item Control de duplicados y estandarización de claves primarias.

\item Creación de agregaciones temporales diarias.
 
\item Creación de variables derivadas de los datos como medias móviles o ratios de inversión.

\item Identificación de picos de campañas.

\item Señalización de eventos especiales tales como: promociones, feriados, lanzamientos.

\item Integración del estado de suscripciones activas e inactivas.

\end{itemize}

Este proceso de transformación se implementó mediante DAGs (\textit{Directed Acyclic Graphs}) en Apache Airflow, lo que permite orquestar de manera programada, reproducible y monitorizable cada etapa del pipeline de datos.

Los datos intermedios y finales de este proceso se almacenan temporalmente en disco, en un formato estructurado y accesible llamado Parquet, para permitir su reutilización en ejecuciones posteriores sin necesidad de reprocesar desde cero. Este enfoque no solo optimiza el uso de recursos, sino que también facilita la depuración y auditoría de los datos generados en cada etapa.

Dado que Shopify y Triple Whale estructuran sus datos de manera diferente, se construyó un esquema interno unificado que respeta la granularidad diaria necesaria para los modelos de predicción. Este esquema queda materializado en una tabla consolidada dentro de PostgreSQL, lista para su consumo por el módulo de modelado.

La implementación mediante Airflow permite además:

\begin{itemize}
\item Ejecución programada o bajo demanda de los flujos de transformación.
\item Reejecución selectiva de tareas fallidas sin afectar etapas exitosas.
\item Monitoreo visual del estado del pipeline y alertas ante incidencias.
\item Versionado y mantenimiento independiente de la lógica de transformación.
\end{itemize}

Este módulo permite encapsular toda la lógica de preparación del dataset, manteniéndolo desacoplado del motor de predicción para que se pueda actualizar fácilmente sin impactar en la generación de pronósticos.


\subsection{Capa de almacenamiento estructurado}

Luego de su procesamiento, los datos se almacenan en una base de datos PostgreSQL diseñada específicamente para consultas analíticas y para servir como fuente unificada de información durante el entrenamiento y evaluación de los modelos. Dentro de esta capa se construye la tabla principal denominada \textit{orders}, en la que se almacena la información proveniente de Shopify ya depurada y enriquecida. Esta tabla contiene las columnas fundamentales para caracterizar el comportamiento de compra de los usuarios:
\begin{itemize}
	\item \textit{created}: registra la fecha de creación de cada orden.
	\item \textit{totalPrice}: correspondiente al monto total pagado.
	\item \textit{customerId}: identifica al cliente.
	\item \textit{lineItems}: donde se detalla cada producto incluido en la compra junto con sus cantidades.
	\item \textit{channel}: indica el canal por el cual ingresó la orden.
	\item \textit{tags}: permite distinguir si la compra corresponde a un cliente nuevo o recurrente mediante etiquetas provistas por Shopify.
	\item \textit{orderNumberForCustomer}: señala el número de orden que representa dentro del historial del cliente (por ejemplo, un valor igual a 1 implica un cliente nuevo).
	\item \textit{diffWeeksFromFirstPurchase}: expresa la cantidad de semanas transcurridas desde la primera compra del usuario, lo que permite analizar patrones de frecuencia y retención.
	
\end{itemize}

A partir de esta tabla estructurada, se generan columnas derivadas mediante procesos de enriquecimiento en Python, con el objetivo de facilitar el análisis y preparar los datos para su uso en modelos predictivos. Entre estas variables se incluyen:
\begin{itemize}
	\item \textit{created\_weekday}: identifica el día de la semana de cada orden.
	\item \textit{created\_month}: permite estudiar estacionalidades mensuales. 
	\item \textit{unique\_customers}: calcula la cantidad de clientes distintos por día.
	\item \textit{new\_customers}: contabiliza cuántos usuarios realizaron su primera compra dentro de cada fecha.
	\item \textit{returning\_customers}: contabiliza cuántos usuarios realizaron más de una compra. 
\end{itemize}

Estas nuevas columnas permiten construir vistas agregadas y analizar la evolución del comportamiento de los clientes en el tiempo.

Finalmente, otra tabla denominada \textit{forecast} consolida la información diaria extraida de TripleWhale. Esta tabla incorpora, entre otras variables, el monto de inversión publicitaria bajo la columna \textit{ad\_spend}. La tabla contiene además otros campos utilizados por el sistema Inventory Tracker, derivados de los datos ya existentes. Sin embargo, dichas columnas no forman parte del alcance del presente trabajo, ya que responden a necesidades operativas específicas del cliente y no intervienen en el proceso de modelado.

Este diseño facilita tanto el acceso para entrenamiento de modelos como la consulta rápida desde Inventory Tracker. En la figura \ref{fig:der} se muestran las dos tablas principales del trabajo con las columnas que se tomaron en cuenta para este proceso de almacenamiento.

\begin{figure}[htbp]
    \centering
    \includegraphics[width=0.8\textwidth]{./Figures/der.png}
    \caption{Diagrama Entidad-Relación. Muestra las dos tablas principales.}
    \label{fig:der}
\end{figure}


\subsection{Capa de modelado predictivo}

Esta capa es responsable del entrenamiento, evaluación y selección de los modelos de pronóstico de ventas. Se implementa un enfoque comparativo donde se ejecutan simultáneamente los modelos mencionados en el capítulo \ref{sec:modelo-ia} (SARIMA, Exponential Smoothing, Random Forest, XGBoost, LightGBM, LSTM y Prophet), con el objetivo de identificar el que ofrezca el mejor desempeño predictivo según las métricas definidas.

El proceso de modelado sigue las siguientes etapas:

\begin{enumerate}
	\item Entrenamiento múltiple: todos los modelos son entrenados utilizando el mismo conjunto de datos históricos consolidado y enriquecido.
	\item Evaluación y métricas: cada modelo es evaluado con la métrica RMSE para penalizar errores grandes que impactan significativamente en la planificación financiera y de producción.

	\item Selección automática: el modelo que obtiene el menor valor de RMSE en el conjunto de validación es seleccionado automáticamente para su uso en producción.
	\item Persistencia del modelo: una vez seleccionado, el modelo ganador es serializado y almacenado en un registro de modelos donde queda disponible para su uso en las predicciones futuras.

	\item Exposición mediante API: el modelo guardado es integrado en un endpoint REST que forma parte del sistema de microservicios para que el sistema Inventory Tracker consulte las predicciones de ventas en tiempo real.
\end{enumerate}








Este diseño no solo asegura que siempre se esté utilizando el modelo más preciso disponible, sino que también facilita la actualización periódica del mismo mediante reentrenamientos programados. Con cada actualización se mantiene la capacidad predictiva del sistema ante cambios en el comportamiento de las ventas o en las dinámicas de mercado.




\subsection{Capa de integración operativa con microservicios}

La arquitectura del sistema se basa en un enfoque de microservicios, donde cada funcionalidad clave se encapsula en un servicio independiente, desplegado y mantenido de forma aislada. Los microservicios implementados son los siguientes:

\begin{itemize}
	\item Servicio de ingestión (cronjobs + conectores API).
	\item Servicio de ETL.
	\item Servicio de predicción.
	\item Servicio de API interna para Inventory Tracker.
\end{itemize}

La comunicación entre estos servicios se realiza mediante endpoints REST internos para operaciones síncronas y, cuando se requiere procesamiento asincrónico o desacoplamiento temporal, a través de colas de mensajería (por ejemplo, con RabbitMQ \citep{rabbitmq} o Redis \citep{redis}). Este diseño permite que cada servicio evolucione y escale de forma independiente, sin afectar al resto del sistema.


Los principales beneficios de esta capa de microservicios son:

\begin{itemize}
	\item Actualizaciones incrementales: cada servicio puede ser actualizado, modificado o reemplazado sin impactar en el funcionamiento de los demás, lo que facilita la incorporación de mejoras o nuevas fuentes de datos.

	\item Escalabilidad selectiva: los módulos con mayor carga computacional, como el servicio de predicción o de ingestión, pueden escalarse horizontalmente de forma independiente, optimizando el uso de recursos.

	\item Aislamiento de fallos: un fallo en un servicio (por ejemplo, la indisponibilidad temporal de una API externa) no propaga su error al resto del sistema, gracias al desacoplamiento y a los mecanismos de tolerancia a fallos implementados.

	\item Facilidad de mantenimiento y despliegue: cada servicio cuenta con su propio repositorio, configuración y ciclo de vida, lo que simplifica las tareas de desarrollo, pruebas y despliegue continuo.
	
\end{itemize}


\subsection{Capa de visualización y consumo dentro de Inventory Tracker}

Las predicciones de ventas finales se integran en el sistema Inventory Tracker mediante una interfaz intuitiva, la cual permite:

\begin{itemize}
	\item Solicitar pronósticos para distintos horizontes temporales.
	\item Visualizar ventas históricas junto con proyecciones.
	\item Combinar la predicción con datos operativos del cliente como: stock, producción, etc.
\end{itemize}

Este componente cierra el ciclo de valor al convertir la información generada por el sistema en una herramienta de apoyo para la gestión diaria.


\subsection{Justificación global de la arquitectura}

La arquitectura fue diseñada bajo los siguientes principios:

\begin{itemize}
	\item Modularidad: permite reemplazar o actualizar partes sin reestructurar toda la solución.

	\item Escalabilidad: preparada para el aumento del volumen de datos y la posible integración de nuevas plataformas.

	\item Trazabilidad: indispensable para un sistema basado en machine learning.

	\item Flexibilidad: permite combinar modelos clásicos, modelos de machine learning y eventualmente servicios administrados en la nube.
\end{itemize}


\section{Automatización mediante tareas programadas}

Para garantizar el funcionamiento continuo y automático del sistema, se diseñó e implementó un conjunto de tareas programadas mediante cronjobs que se ejecutan de manera secuencial y periódica, sin requerir intervención manual. Este flujo automatizado permite mantener la información del sistema actualizada de forma confiable y oportuna. Las principales tareas programadas incluyen:

\begin{itemize}
    \item Extracción diaria de datos: se ejecuta una conexión automatizada a las APIs de Shopify y Triple Whale para obtener los datos más recientes de ventas e inversión publicitaria. Este proceso garantiza que el sistema opere con información actualizada y evita retrasos en la disponibilidad de datos.
    
    \item Actualización del dataset consolidado: los nuevos datos se procesan, limpian y unifican con el historial existente. Este paso genera un dataset integral y consistente que sirve como base para el entrenamiento del modelo y la generación de pronósticos.
    
    \item Reentrenamiento periódico del modelo: con una frecuencia configurada (por ejemplo, semanal o mensual), el sistema ejecuta automáticamente el proceso de reentrenamiento del modelo de pronóstico utilizando el dataset actualizado. Esto permite que el modelo se adapte a cambios en las tendencias de ventas, comportamiento del mercado o nuevos patrones estacionales.
    
    \item Generación de nuevas predicciones: después del reentrenamiento, el modelo genera pronósticos actualizados para los períodos futuros definidos. Estas predicciones reemplazan a las versiones anteriores y se almacenan en la base de datos para su consulta.
    
    \item Invalidación y regeneración de caché en Inventory Tracker: Para asegurar que los usuarios accedan siempre a la información más reciente, el sistema invalida automáticamente la caché del Inventory Tracker y la regenera con las nuevas predicciones y datos consolidados. Esto elimina posibles desfases entre los datos mostrados y la realidad del sistema.
\end{itemize}

Adicionalmente, se implementaron mecanismos de monitoreo y registro (\textit{logging}) para cada tarea programada, lo que permite auditar el correcto funcionamiento del flujo automatizado y detectar posibles fallos de manera temprana. 

Esta automatización integral asegura que el cliente disponga permanentemente de información actualizada, confiable y lista para la toma de decisiones, mientras se reduce significativamente el riesgo de errores humanos y se optimiza el mantenimiento operativo del sistema.


\section{Análisis de datos}
\section{Desarrollo de modelos}
\section{Desarrollo de un framework modular}
\section{Despliegue con microservicios e integración con bases de datos}

\begin{figure}[htbp]
	\centering
	\includegraphics[width=1\textwidth]{./Figures/diagBloques.png}
	\caption{Flujo general del sistema para generar el pronóstico de ventas.}

	\label{fig:texmaker}
\end{figure}

% Chapter Template

\chapter{Ensayos y resultados} % Main chapter title

\label{Chapter4} % Change X to a consecutive number; for referencing this chapter elsewhere, use \ref{ChapterX}
Todos los capítulos deben comenzar con un breve párrafo introductorio que indique cuál es el contenido que se encontrará al leerlo.  La redacción sobre el contenido de la memoria debe hacerse en presente y todo lo referido al proyecto en pasado, siempre de modo impersonal.

%----------------------------------------------------------------------------------------
%	SECTION 1
%----------------------------------------------------------------------------------------

\section{Ensayos de modelos}
\label{sec:pruebasHW}

La idea de esta sección es explicar cómo se hicieron los ensayos, qué resultados se obtuvieron y analizarlos.

\section{Desempeño del modelo}
\section{Desempeño de la predicción}
\section{Comparación de algoritmos}
\section{Validación de cumplimiento de requerimientos}

A continuación se detalla la descripción técnica y alcance correspondientes a cada uno de los requerimientos especificados anteriormente.


\subsubsection{Requerimiento funcional 1}

Para cumplir este requerimiento, se desarrolló un módulo de integración que:

\begin{itemize}
    \item Establece conexión con la API de Shopify mediante credenciales \textit{API Key} y \textit{Access Token}.
    \item Obtiene y procesa los siguientes atributos mínimos: fecha de compra, identificador de producto, cantidad vendida, precio unitario y canal de venta.
    \item Permite especificar intervalos de fechas para la descarga de datos.
    \item Almacena los datos importados en una base de datos PostgreSQL destinada a su posterior uso en el modelado.
    \item Implementa validaciones y manejo de errores en caso de fallos de autenticación, pérdida de conexión o respuestas vacías.
    \item Registra la trazabilidad del proceso de importación mediante un archivo de log, incluyendo volumen de datos procesados y errores detectados.
    \item Verifica duplicidades para evitar el doble procesamiento de registros previamente importados.
\end{itemize}

\subsubsection{Requerimiento funcional 2}

Para satisfacer este requerimiento se desarrolló un módulo de integración que:

\begin{itemize}
    \item Establece conexión con la API de Triple Whale con una clave de acceso provista por la organización.
    \item Recupera la información de inversión por fecha, campaña y plataforma publicitaria.
    \item Permite seleccionar y configurar el intervalo temporal de los datos a importar.
    \item Almacena los datos obtenidos en una base de datos PostgreSQL, manteniendo una estructura consistente con el modelo de análisis utilizado.
    \item Implementa mecanismos de manejo de errores ante fallos de autenticación, interrupciones en la conexión o respuestas inválidas provenientes de la API.
    \item Registra el detalle del proceso mediante un archivo de log, incluyendo la cantidad de registros importados y los errores detectados durante la ejecución.
    \item Permite ejecutar el proceso de importación tanto de forma manual como programada mediante tareas periódicas.
    \item Incluye validaciones para evitar la duplicación de registros en caso de importaciones sobre el mismo rango de fechas.
\end{itemize}



\subsubsection{Requerimiento funcional 3}

Para satisfacer este requerimiento se desarrolló un proceso de consolidación que integra los datos provenientes de Shopify (ventas) y Triple Whale (inversión publicitaria), y garantiza la correcta correspondencia temporal entre ambas fuentes. En particular:

\begin{itemize}
    \item Se generó un único \textit{dataset} en el que las ventas diarias y la inversión total diaria se vinculan a través del campo de fecha.
    \item Se realizó el alineamiento temporal de las fechas, y se tuvo en cuenta posibles diferencias de zona horaria entre ambas plataformas.
    \item En los casos en que alguna de las fuentes no presentaba información para una fecha determinada, se mantuvo el registro correspondiente utilizando valores nulos para evitar la pérdida de información histórica.
    \item El conjunto de datos final incluye las siguientes columnas: \texttt{fecha},  \texttt{ventas\_diarias}, \texttt{inversion\_diaria} y \texttt{plataforma}.
    \item Se aplicaron validaciones para detectar y corregir duplicados, inconsistencias o formatos inválidos en los registros procesados.
    \item El proceso de consolidación se diseñó de manera automatizada y reproducible para su ejecución periódica sin intervención manual.
    \item El \textit{dataset} resultante se almacenó en un formato estructurado apto para ser consumido directamente en las etapas de modelado y entrenamiento.
    \item Se registraron las operaciones del proceso mediante archivos de log, incluyendo la cantidad total de registros consolidados y cualquier incidencia detectada durante la ejecución.
\end{itemize}



\subsubsection{Requerimiento funcional 4}


Para cumplir con este requerimiento se implementó un proceso de modelado y entrenamiento estructurado que incluyó las siguientes etapas:

\begin{itemize}
    \item Se empleó el \textit{dataset} consolidado, compuesto por variables de ventas y de inversión publicitaria, como fuente principal para el entrenamiento del modelo.
    \item El conjunto de datos fue dividido en subconjuntos de entrenamiento y validación, siguiendo una proporción aproximada de 80\% para entrenamiento y 20\% para validación.
    \item Se evaluaron distintos modelos de predicción utilizando métricas de error adecuadas para series temporales, tales como MAE, RMSE y MAPE, con el fin de comparar su desempeño.
    \item Se seleccionó como modelo final aquel que presentó el mejor desempeño sobre el conjunto de validación.
    \item El modelo entrenado quedó habilitado para generar predicciones futuras a partir de valores recientes de entrada.
    \item Se garantizó la reproducibilidad del proceso mediante la definición de un \textit{pipeline} de entrenamiento que permite repetir la ejecución bajo las mismas configuraciones.
    \item El modelo seleccionado y sus parámetros finales fueron almacenados para su posterior uso dentro del sistema Inventory Tracker.
    \item Se documentó el proceso de entrenamiento de manera detallada, incluyendo la selección de variables, las técnicas de preprocesamiento aplicadas y los métodos de evaluación utilizados.
    \item Se realizó una validación visual comparando las predicciones generadas frente a los valores reales observados, mediante gráficos que permitieron analizar la coherencia del modelo respecto a la dinámica histórica de las ventas.
\end{itemize}


\subsubsection{Requerimiento funcional 5}

Para cumplir con este requerimiento se implementaron las siguientes funcionalidades:

\begin{itemize}
    \item Las predicciones de ventas se muestran en la interfaz de Inventory Tracker junto a las fechas correspondientes.
    \item La visualización integra tanto las predicciones como los datos históricos de ventas para comparaciones temporales y análisis de tendencias.
    \item La información se presenta de manera clara y entendible, utilizando gráficos de líneas y tablas según corresponda.
    \item Los usuarios pueden seleccionar rangos de fechas específicos para observar predicciones detalladas de periodos concretos.
    \item Las predicciones se actualizan automáticamente al entrenarse un nuevo modelo o al actualizarse los datos de entrada y se garantiza que la información refleje siempre el estado más reciente.
    \item La interfaz muestra un indicador visual de la fecha y hora de generación de la última predicción.
    \item Se distingue claramente entre valores predichos y datos históricos que evitan confusiones en el análisis.
    \item La integración de las predicciones no afecta la funcionalidad ni el rendimiento general de la interfaz del módulo Inventory Tracker.
    \item El rendimiento de la interfaz se mantiene aceptable incluso cuando se manejan conjuntos de datos de gran tamaño.
\end{itemize}


\subsubsection{Requerimiento funcional 6}


Para cumplir con este requerimiento se implementaron las siguientes funcionalidades:

\begin{itemize}
    \item Se incorporó un botón o enlace claramente identificado que permite la descarga de las predicciones en formato CSV.
    \item El archivo CSV generado incluye las fechas y los valores de predicción correspondientes.
    \item De manera opcional, el archivo puede incluir los datos históricos de ventas para permitir la comparación con las predicciones.
    \item El formato del CSV es compatible con herramientas comunes de análisis, con valores separados por comas y codificación UTF-8.
    \item El nombre del archivo contiene la fecha y hora de generación que lo identifica.
    \item La descarga se realiza correctamente en navegadores modernos, incluyendo Chrome y Firefox.
    \item El contenido del archivo refleja exactamente la información presentada en la interfaz del módulo de predicción.
    \item Si aún no existen predicciones generadas, el botón de descarga se muestra deshabilitado.
\end{itemize}

\subsubsection{Requerimiento funcional 7}

Para cumplir con este requerimiento se implementaron las siguientes funcionalidades:

\begin{itemize}
    \item El sistema verifica automáticamente la integridad del \textit{dataset} antes de cada entrenamiento del modelo.
    \item Si se detectan datos faltantes críticos como fechas sin registros de ventas o de inversión publicitaria se genera una alerta.
    \item La alerta incluye información detallada sobre el tipo de dato faltante (ventas, inversión, o ambos), el rango de fechas afectado y el nivel de severidad del impacto estimado.
    \item La alerta se muestra de forma visible dentro del sistema para que los responsables puedan tomar conocimiento inmediato.
    \item Se envía una notificación por correo electrónico al responsable del sistema, en caso de estar configurada esta opción.
    \item El sistema no bloquea el entrenamiento del modelo, pero advierte que la precisión de las predicciones puede verse afectada.
    \item La alerta desaparece únicamente cuando los datos faltantes son completados o cuando se marca como revisada manualmente.
    \item Se registra un log de todas las alertas emitidas, accesible desde una sección de administración o monitoreo del sistema.
\end{itemize}


\include{Chapters/Chapter5}

%----------------------------------------------------------------------------------------
% Apéndices
%----------------------------------------------------------------------------------------

\appendix

% Incluir apéndices desde archivos separados si es necesario
%\include{Appendices/AppendixA}

%----------------------------------------------------------------------------------------
% Bibliografía
%----------------------------------------------------------------------------------------

\renewcommand{\bibname}{Bibliografía} % Para asegurarte de que el título sea correcto
\phantomsection % Necesario para que el enlace del marcador sea correcto

\printbibliography[heading=bibintoc]

\end{document}






