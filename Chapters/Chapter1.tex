% Chapter 1

\chapter{Introducción general} % Main chapter title

\label{Chapter1} % For referencing the chapter elsewhere, use \ref{Chapter1} 
\label{IntroGeneral}

En este capítulo se presenta el contexto general del trabajo y la problemática que motivó su desarrollo. Se exponen las razones y necesidades que impulsaron su realización, junto con una descripción de las soluciones existentes y los enfoques actuales relacionados con la temática abordada. Asimismo, se detallan los propósitos principales del trabajo y se delimitan los alcances y límites de su implementación.

%----------------------------------------------------------------------------------------

% Define some commands to keep the formatting separated from the content 
\newcommand{\keyword}[1]{\textbf{#1}}
\newcommand{\tabhead}[1]{\textbf{#1}}
\newcommand{\code}[1]{\texttt{#1}}
\newcommand{\file}[1]{\texttt{\bfseries#1}}
\newcommand{\option}[1]{\texttt{\itshape#1}}
\newcommand{\grados}{$^{\circ}$}

%----------------------------------------------------------------------------------------
\section{Contexto del Trabajo}

El presente trabajo forma parte de la Carrera de Especialización en Inteligencia Artificial y tiene como propósito el desarrollo de un modelo de inteligencia artificial capaz de pronosticar las ventas de una tienda en línea perteneciente a la empresa Latech.
La empresa fabrica y comercializa barritas alimenticias en distintos sabores, lo que requiere una planificación de producción precisa para evitar tanto faltantes como excedentes de stock. En este contexto, contar con pronósticos confiables de ventas constituye un factor estratégico clave para la toma de decisiones operativas y comerciales.

El trabajo se desarrolló con financiamiento de la empresa Latech y tuvo un presupuesto estimado de 684 horas y un costo aproximado de \$12.825.000, con fecha de inicio el 24 de junio de 2025 y presentación pública prevista para abril de 2026.

Para la elaboración del modelo se dispuso del acceso a la API de Shopify, plataforma en la nube que aloja la tienda del cliente, y a la API de Triple Whale, una herramienta especializada en la gestión y análisis de inversión publicitaria. Estas fuentes de datos permitieron obtener información histórica de aproximadamente dos años sobre ventas e inversión en publicidad digital.


\begin{figure}[htbp]
	\centering
	\includegraphics[width=1\textwidth]{./Figures/diagBloques.png}
	\caption{Diagrama en bloques.}
	\label{fig:texmaker}
\end{figure}


El sistema desarrollado se integró dentro del entorno interno del cliente, denominado Inventory Tracker, a fin de generar pronósticos de ventas a partir de los datos recopilados. En la Figura 1 se presenta el diagrama en bloques del sistema, donde se observa el flujo de información desde las APIs externas hasta la generación de los resultados.


\section{Motivación}

El desarrollo de este trabajo surge de la necesidad de la empresa Latech de disponer de una herramienta que le permita anticipar el comportamiento de las ventas de su tienda en línea y, de esta manera, mejorar la planificación de la producción y la gestión del stock.

La falta de un sistema de pronóstico confiable generaba dificultades en la toma de decisiones estratégicas y operativas, esto afectaba tanto la disponibilidad de productos como la eficiencia de las campañas de marketing. Por ello, se consideró necesario diseñar una solución basada en técnicas de inteligencia artificial que proporcionara predicciones con un alto grado de precisión y permitiera analizar la evolución de las ventas en el tiempo.

Asimismo, se buscó analizar el comportamiento de los usuarios. Se observó la cantidad de clientes recurrentes y los canales de venta utilizados. Los datos indicaron que las compras se realizaban a través de diferentes plataformas: Skio, un servicio que gestiona pagos y ventas adicionales en la tienda; TikTok, Facebook, Instagram y la propia tienda alojada en Shopify. Esta información resultó fundamental para comprender los patrones de compra y optimizar las predicciones del modelo.

Además, el uso de fuentes de datos integradas, como Shopify y Triple Whale, posibilitó aprovechar información histórica de ventas, inversión publicitaria y comportamiento de los usuarios, lo que resultó esencial para la construcción de un modelo predictivo robusto y adaptable a las necesidades del cliente.

En este sentido, el trabajo buscó no solo resolver una problemática específica de la empresa, sino también demostrar el potencial del aprendizaje automático aplicado al comercio electrónico como herramienta de apoyo a la toma de decisiones empresariales.



\section{Estado del arte}

En la actualidad, el pronóstico de ventas mediante técnicas de inteligencia artificial constituye un área de creciente interés en el ámbito del comercio electrónico. Los avances en el análisis de datos y en el aprendizaje automático han permitido el desarrollo de modelos capaces de anticipar la demanda con altos niveles de precisión. Esto contribuye a una mejor planificación de la producción, la gestión de inventarios y la toma de decisiones estratégicas.

Entre los enfoques más utilizados se encuentran los modelos de series temporales clásicos, como ARIMA, SARIMA y \textit{Exponential Smoothing}, que permiten capturar patrones estacionales y tendencias a lo largo del tiempo. Sin embargo, estos métodos suelen presentar limitaciones cuando los datos incluyen múltiples factores externos, como campañas publicitarias, canales de venta o comportamiento del usuario.

En respuesta a esas limitaciones, en los últimos años se ha incrementado el uso de técnicas de aprendizaje profundo (\textit{Deep Learning}) y aprendizaje automático (\textit{Machine Learning}), que ofrecen una mayor capacidad para modelar relaciones no lineales y para integrar múltiples fuentes de información. Modelos como Redes Neuronales Recurrentes (RNN), LSTM (\textit{Long Short-Term Memory}) y \textit{Random Forest Regressor} se han aplicado con éxito en la predicción de ventas en entornos de comercio electrónico, debido a su habilidad para capturar dependencias temporales y correlaciones entre variables.

Asimismo, el uso de plataformas integradas de datos, tales como Shopify y Triple Whale, ha impulsado el desarrollo de soluciones personalizadas que combinan información transaccional, métricas de marketing y comportamiento de los usuarios. Estas herramientas permiten no solo mejorar la precisión de los modelos de pronóstico, sino también ofrecer una visión integral del desempeño comercial y de la efectividad de las campañas publicitarias.

Entre las soluciones comerciales más reconocidas se destacan Amazon Forecast, un servicio basado en redes neuronales desarrollado por Amazon Web Services que automatiza la creación de modelos de predicción de demanda, y Prophet, una herramienta de código abierto creada por Meta (Facebook) que utiliza un enfoque aditivo para modelar tendencias y estacionalidades de manera flexible. Ambas herramientas representan referencias relevantes en el campo del pronóstico de series temporales con datos de negocios.

Diversos estudios recientes también han explorado la aplicación de modelos híbridos que combinan técnicas tradicionales y redes neuronales para mejorar la precisión de los resultados. Estos enfoques se han implementado especialmente en sectores minoristas, donde la influencia de factores externos —como la publicidad digital, las promociones y la estacionalidad— tiene un peso significativo en la dinámica de ventas.

En este marco, el trabajo desarrollado se apoyó en los enfoques actuales de predicción de demanda mediante aprendizaje automático, incorporando datos históricos de ventas, inversión publicitaria y comportamiento del usuario. El objetivo fue construir un modelo que reflejara las tendencias reales del negocio y sirviera como herramienta de apoyo para la toma de decisiones estratégicas en la empresa Latech.


\section{Objetivo y Alcance}

En esta sección se mencionan los propósitos principales del trabajo y los límites de su implementación.

\subsection{Objetivo del trabajo}

El propósito de este trabajo es desarrollar un modelo de inteligencia artificial que permita predecir con precisión las ventas del producto alimenticio de la empresa Latech, a partir de datos de ventas y de inversión publicitaria. Esto permitirá a la empresa: anticipar sus necesidades de
producción, optimizar la planificación del inventario y tomar decisiones estratégicas basadas
en datos. Asimismo, podrá reducir el riesgo de faltantes o excesos de inventario, y mejorar la
eficiencia operativa y comercial.

\subsection{Alcance del trabajo}

A continuación, se enuncia los items que incluye el trabajo:

\begin{itemize}
    \item Relevamiento y análisis de los datos históricos de ventas de la tienda en línea obtenidos mediante la API de Shopify.
    
    \item Relevamiento y análisis de los datos históricos de inversión publicitaria provenientes de la
API de Triple Whale.

    \item Limpieza, transformación y consolidación de datos provenientes de ambas plataformas.
    
     \item Análisis exploratorio de datos (EDA) para identificar patrones, tendencias y relaciones
entre las variables.
     
     \item Desarrollo y entrenamiento de modelos de inteligencia artificial orientados a la predicción
de ventas.
     
     \item Evaluación comparativa de distintos modelos para seleccionar el que ofrezca la mejor
precisión y capacidad predictiva.
     
     \item Implementación de un módulo funcional integrado al sistema Inventory Tracker de la
empresa, que permita realizar predicciones de ventas para períodos futuros.

     
     \item Documentación técnica del proceso, del modelo elegido y de su uso.
     
     \item Entrega de reportes que expliquen los hallazgos y recomendaciones derivadas del análisis.
     
\end{itemize}

A modo de limitaciones en la implementación se enuncian loa items que no incluye el trabajo:

\begin{itemize}
    \item Implementaciones en tiempo real del modelo, es decir, sistemas que actualicen las
predicciones de forma instantánea ante cada nuevo dato recibido. Sí se contempla la
posibilidad de programar ejecuciones periódicas del modelo (por ejemplo, una vez al día)
para actualizar las predicciones de manera regular.
	\item Garantía de precisión absoluta en las predicciones, ya que la precisión depende de la
calidad y estabilidad de los datos futuros y de factores externos no controlables.
	\item Acciones o recomendaciones específicas sobre estrategias de marketing más allá de lo
inferido de los análisis de datos.

	\item Optimización de procesos internos de producción o logística, salvo en lo que respecta a la estimación de demanda.

\end{itemize}

\subsection{Condiciones del trabajo}

El desarrollo del presente trabajo se realizó considerando las siguientes condiciones:

\begin{itemize}
    \item La empresa Latech proporcionó acceso completo y continuo a las APIs externas de Shopify y Triple Whale, lo que permitió obtener datos históricos y actualizados.

    \item Se contó con acceso al repositorio de código y a la infraestructura necesaria para integrar el módulo de predicción en el sistema existente Inventory Tracker.

    \item La calidad, integridad y consistencia de los datos obtenidos desde las plataformas externas fue suficiente para entrenar y validar los modelos de predicción.

    \item Durante el período de desarrollo no se produjeron cambios significativos en las políticas de acceso ni en la estructura de datos de las APIs utilizadas.

    \item No existieron restricciones legales o contractuales que impidieran el uso de los datos necesarios para el trabajo.

\end{itemize}
