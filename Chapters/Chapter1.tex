% Chapter 1

\chapter{Introducción general} % Main chapter title

\label{Chapter1} % For referencing the chapter elsewhere, use \ref{Chapter1} 
\label{IntroGeneral}

En este capítulo se presenta el contexto general del trabajo y la problemática que motivó su desarrollo. Se exponen las razones y necesidades que impulsaron su realización, junto con una descripción de las soluciones existentes y los enfoques actuales relacionados con la temática abordada. Asimismo, se detallan los propósitos principales del trabajo y se delimitan los alcances y límites de su implementación.

%----------------------------------------------------------------------------------------

% Define some commands to keep the formatting separated from the content 
\newcommand{\keyword}[1]{\textbf{#1}}
\newcommand{\tabhead}[1]{\textbf{#1}}
\newcommand{\code}[1]{\texttt{#1}}
\newcommand{\file}[1]{\texttt{\bfseries#1}}
\newcommand{\option}[1]{\texttt{\itshape#1}}
\newcommand{\grados}{$^{\circ}$}

%----------------------------------------------------------------------------------------
\section{Contexto y motivación}

El presente trabajo forma parte de la Carrera de Especialización en Inteligencia Artificial y tiene como propósito el desarrollo de un modelo capaz de pronosticar las ventas de una tienda en línea perteneciente a la empresa Latech.
La empresa fabrica y comercializa barritas alimenticias en distintos sabores, lo que requiere una planificación de producción precisa para evitar tanto faltantes como excedentes de stock. En este contexto, contar con pronósticos confiables de ventas constituye un factor estratégico clave para la toma de decisiones operativas y comerciales.

La ausencia de un sistema de pronóstico confiable generaba dificultades en la planificación estratégica y operativa de la empresa. En particular, cuando una campaña de marketing resultaba exitosa y aumentaban significativamente las visitas y las compras, la capacidad de producción no siempre lograba adaptarse con la suficiente rapidez, lo que en ocasiones derivaba en falta de stock. Esta situación ya se había presentado previamente y había generado retrasos en la entrega de pedidos y afectado la experiencia del cliente.


\begin{figure}[htbp]
	\centering
	\includegraphics[width=1\textwidth]{./Figures/diagBloques.png}
	\caption{Flujo general del sistema para generar el pronóstico de ventas.}

	\label{fig:texmaker}
\end{figure}

La integración de las APIs permite centralizar información obtenida desde diversos canales de difusión. Entre ellas se encuentran TikTok, Instagram, Facebook y Google. Esta diversidad de fuentes permite identificar qué plataformas participan en el proceso de conversión y servirá como base para estimar aspectos como la proporción de usuarios nuevos y recurrentes, así como las variaciones en el volumen de ventas según el canal.
Además, en el sitio de ventas los usuarios pueden suscribirse a compras mensuales, lo que permite analizar la actividad de las suscripciones: identificar cuándo los usuarios mantienen sus suscripciones activas y cuándo las cancelan. Esta información adicional resulta valiosa para comprender patrones de compra recurrentes y estimar la demanda futura de manera más precisa.

La incorporación conjunta de datos históricos de ventas (Shopify) y de inversión publicitaria (Triple Whale) proporciona un insumo clave para el entrenamiento del modelo de predicción de ventas. Contar con estas dos fuentes permite relacionar la inversión realizada en cada plataforma con los resultados obtenidos en términos de ventas, lo que contribuye a la construcción de un modelo más preciso y contextualizado.

De este modo, el trabajo apunta a desarrollar un sistema que utilice aprendizaje automático como herramienta de apoyo a la planificación comercial y a la toma de decisiones basadas en datos.



\section{Estado del arte}

En la actualidad, el pronóstico de ventas mediante técnicas de inteligencia artificial constituye un área de creciente interés en el ámbito del comercio electrónico. Los avances en el análisis de datos y en el aprendizaje automático han permitido el desarrollo de modelos capaces de anticipar la demanda con altos niveles de precisión. Esto contribuye a una mejor planificación de la producción, la gestión de inventarios y la toma de decisiones estratégicas.

Entre los enfoques más utilizados se encuentran los modelos de series temporales clásicos, como ARIMA \citep{BOOK:4}, SARIMA y \textit{Exponential Smoothing}, que permiten capturar patrones estacionales y tendencias a lo largo del tiempo \citep{BOOK:3}. Sin embargo, estos métodos suelen presentar limitaciones cuando los datos incluyen múltiples factores externos, como campañas publicitarias, canales de venta o comportamiento del usuario.

En respuesta a esas limitaciones, en los últimos años se ha incrementado el uso de técnicas de aprendizaje profundo (\textit{Deep Learning}) y aprendizaje automático (\textit{Machine Learning}), que ofrecen una mayor capacidad para modelar relaciones no lineales y para integrar múltiples fuentes de información. Modelos como Redes Neuronales Recurrentes (RNN), LSTM (\textit{Long Short-Term Memory}) y \textit{Random Forest Regressor} se han aplicado con éxito en la predicción de ventas en entornos de comercio electrónico, debido a su habilidad para capturar dependencias temporales y correlaciones entre variables \cite{hochreiter1997lstm, goodfellow2016deep, breiman2001random, bandara2020forecast}.

Asimismo, el uso de plataformas integradas de datos, tales como Shopify \citep{shopify} y Triple Whale \citep{triplewhale}, ha impulsado el desarrollo de soluciones personalizadas que combinan información transaccional, métricas de marketing y comportamiento de los usuarios. Estas herramientas permiten no solo mejorar la precisión de los modelos de pronóstico, sino también ofrecer una visión integral del desempeño comercial y de la efectividad de las campañas publicitarias.

Entre las soluciones comerciales más reconocidas se destacan Amazon Forecast \citep{amazonforecast}, un servicio basado en redes neuronales desarrollado por Amazon Web Services \citep{aws} que automatiza la creación de modelos de predicción de demanda, y Prophet \citep{taylor2018forecasting}, una herramienta de código abierto creada por Meta (Facebook) \citep{meta} que utiliza un enfoque aditivo para modelar tendencias y estacionalidades de manera flexible. Ambas herramientas representan referencias relevantes en el campo del pronóstico de series temporales con datos de negocios.

La tendencia actual en el campo apunta hacia el desarrollo de modelos híbridos, que combinan la capacidad de las redes neuronales para capturar relaciones no lineales y complejas con la robustez e interpretabilidad de las técnicas tradicionales (como ARIMA o SARIMA). Esta combinación permite alcanzar una mayor precisión y robustez en los pronósticos. El sector minorista ha sido el principal adoptante de estos enfoques, ya que sus ventas están muy influenciadas por factores externos como la publicidad digital, las promociones y la estacionalidad.

En este marco, el trabajo desarrollado se apoyó en los enfoques actuales de predicción de demanda mediante aprendizaje automático, datos históricos de ventas, inversión publicitaria y comportamiento del usuario. El objetivo fue construir un modelo que reflejara las tendencias reales del negocio y sirviera como herramienta de apoyo para la toma de decisiones estratégicas en la empresa Latech.


\section{Objetivo y alcance}

En esta sección se mencionan los propósitos principales del trabajo y los límites de su implementación.

\subsection{Objetivo del trabajo}

El propósito de este trabajo es desarrollar un modelo de inteligencia artificial que permita predecir con precisión las ventas del producto alimenticio de la empresa Latech, a partir de datos de ventas y de inversión publicitaria. Esto permitirá a la empresa: anticipar sus necesidades de
producción, optimizar la planificación del inventario y tomar decisiones estratégicas basadas
en datos. Asimismo, podrá reducir el riesgo de faltantes o excesos de inventario, y mejorar la
eficiencia operativa y comercial.

\subsection{Alcance del trabajo}

A continuación, se enuncian los items que incluye el trabajo:

\begin{itemize}
    \item Relevamiento y análisis de los datos históricos de ventas de la tienda en línea obtenidos mediante la API de Shopify.
    
    \item Relevamiento y análisis de los datos históricos de inversión publicitaria provenientes de la
API de Triple Whale.

    \item Limpieza, transformación y consolidación de datos provenientes de ambas plataformas.
    
     \item Análisis exploratorio de datos (EDA) para identificar patrones, tendencias y relaciones
entre las variables.
     
     \item Desarrollo y entrenamiento de modelos de inteligencia artificial orientados a la predicción
de ventas.
     
     \item Evaluación comparativa de distintos modelos para seleccionar el que ofrezca la mejor
precisión y capacidad predictiva.
     
     \item Implementación de un módulo funcional integrado al sistema Inventory Tracker de la
empresa, que permita realizar predicciones de ventas para períodos futuros.

     
     \item Documentación técnica del proceso, del modelo elegido y de su uso.
     
     \item Entrega de reportes que expliquen los hallazgos y recomendaciones derivadas del análisis.
     
\end{itemize}

Por otra parte, quedan excluidas del alcance las siguientes actividades:

\begin{itemize}
    \item Implementaciones en tiempo real del modelo, es decir, sistemas que actualicen las
predicciones de forma instantánea ante cada nuevo dato recibido. Sí se contempla la
posibilidad de programar ejecuciones periódicas del modelo (por ejemplo, una vez al día)
para actualizar las predicciones de manera regular.
	\item Garantía de precisión absoluta en las predicciones, ya que la precisión depende de la
calidad y estabilidad de los datos futuros y de factores externos no controlables.
	\item Acciones o recomendaciones específicas sobre estrategias de marketing más allá de lo
inferido de los análisis de datos.

	\item Optimización de procesos internos de producción o logística, salvo en lo que respecta a la estimación de demanda.

\end{itemize}

