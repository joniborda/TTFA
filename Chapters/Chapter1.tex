% Chapter 1

\chapter{Introducción general} % Main chapter title

\label{Chapter1} % For referencing the chapter elsewhere, use \ref{Chapter1} 
\label{IntroGeneral}

En este capítulo se presenta el contexto general del trabajo y la problemática que motivó su desarrollo. Se exponen las razones y necesidades que impulsaron su realización, junto con una descripción de las soluciones existentes y los enfoques actuales relacionados con la temática abordada. Asimismo, se detallan los propósitos principales del trabajo y se delimitan los alcances y límites de su implementación.

%----------------------------------------------------------------------------------------

% Define some commands to keep the formatting separated from the content 
\newcommand{\keyword}[1]{\textbf{#1}}
\newcommand{\tabhead}[1]{\textbf{#1}}
\newcommand{\code}[1]{\texttt{#1}}
\newcommand{\file}[1]{\texttt{\bfseries#1}}
\newcommand{\option}[1]{\texttt{\itshape#1}}
\newcommand{\grados}{$^{\circ}$}

%----------------------------------------------------------------------------------------
\section{Contexto y motivación}

El presente trabajo forma parte de la Carrera de Especialización en Inteligencia Artificial y tiene como propósito el desarrollo de un modelo capaz de pronosticar las ventas de una tienda en línea perteneciente a la empresa Latech. Esta compañía fabrica y comercializa barritas alimenticias en distintos sabores, lo que exige una planificación de producción precisa para evitar tanto faltantes como excedentes de stock. En este contexto, disponer de pronósticos confiables de ventas representa un factor estratégico para las decisiones operativas y comerciales.

La empresa ofrece siete sabores considerados productos base, disponibles de forma permanente, y además introduce ediciones limitadas que agregan entre dos y cuatro variantes nuevas por año. Estas ediciones dependen del nivel de aceptación del público y permanecen activas solo mientras alcanzan un volumen de ventas satisfactorio. El producto se vende en paquetes de 10, 20 y 30 unidades, con un precio por unidad que disminuye a medida que aumenta el tamaño del paquete. Asimismo, el sitio ofrece al usuario la opción de suscribirse, lo que permite recibir el producto mensualmente con un descuento aproximado del 10\%. Este sistema facilita la construcción de patrones de demanda asociados a consumos recurrentes y posibilita el análisis del ciclo de vida de las suscripciones, incluyendo altas, bajas y reactivaciones.

Latech lanza cada nueva variante de sabor mediante campañas simultáneas en sus canales de comunicación, como TikTok \citep{tiktokAds}, Instagram \citep{InstagramAds}, Facebook \citep{FacebookAds} y Google \citep{GoogleAds}. Estos lanzamientos producen un aumento significativo en la actividad comercial y suelen duplicar las ventas de la nueva variante en comparación con los productos restantes. Un comportamiento similar aparece durante eventos promocionales, como \textit{Black Friday}, donde la aplicación de descuentos generales genera un incremento promedio cercano al 20\% en las ventas totales. Estas fluctuaciones introducen picos marcados en la demanda, cuya anticipación resulta crucial para coordinar producción, logística y abastecimiento.

Antes del desarrollo de este sistema, la empresa no contaba con un mecanismo de pronóstico capaz de anticipar estos cambios de manera sistemática. Las variaciones abruptas en la demanda, especialmente las derivadas de campañas exitosas, superaban en ocasiones la capacidad inmediata de producción. En consecuencia, la organización registraba episodios de quiebre de stock, retrasos en las entregas y deterioro en la experiencia del cliente. La falta de previsión reducía la capacidad de planificar compras de insumos, organizar turnos de producción y definir niveles óptimos de inventario.

La incorporación de datos provenientes de múltiples fuentes permite construir una visión integral del proceso comercial. 
La empresa utiliza la API de Triple Whale \citep{triplewhale} para concentrar la información de inversión publicitaria de TikTok, Instagram, Facebook y Google.
Esta información permite identificar qué canales influyen en la conversión, la proporción de usuarios nuevos y recurrentes, 
y las variaciones en el volumen de ventas según la estrategia de difusión. 
Por su parte, los datos históricos de ventas extraídos desde Shopify \citep{shopify} permiten reconstruir la evolución temporal de la demanda, medir el impacto de las ediciones limitadas, cuantificar el efecto de las promociones y analizar patrones estacionales. La combinación de ambas fuentes constituye un insumo esencial para entrenar un modelo de predicción de ventas robusto y contextualizado.

En conjunto, este trabajo apunta a desarrollar un sistema basado en aprendizaje automático que proporcione estimaciones confiables de demanda y que funcione como herramienta de apoyo a la planificación comercial. Su objetivo final consiste en mejorar la toma de decisiones en áreas como producción, abastecimiento, definición de inventarios, activación de campañas y análisis del comportamiento del cliente, que favorezca una gestión más eficiente y orientada a datos.


\section{Estado del arte}

En la actualidad, el pronóstico de ventas mediante técnicas de inteligencia 
artificial constituye un área de creciente interés en el ámbito del comercio electrónico. 
Los avances en el análisis de datos y en el aprendizaje automático han permitido 
la aparición de modelos capaces de anticipar la demanda con altos niveles de 
precisión, lo que contribuye a una mejor planificación de la producción,
 la gestión de inventarios y la toma de decisiones estratégicas.

Entre los enfoques más utilizados se encuentran los modelos de series temporales
 clásicos, como ARIMA \citep{BOOK:4}, SARIMA y \textit{Exponential Smoothing}, 
 que permiten capturar patrones estacionales y tendencias a lo largo del tiempo
  \citep{BOOK:3}. Sin embargo, estos métodos presentan limitaciones cuando los 
  datos incluyen múltiples factores externos, tales como campañas publicitarias,
   variaciones en los precios, promociones o comportamiento del usuario.

En los últimos años se ha observado un aumento significativo en la adopción de
 técnicas de aprendizaje profundo (\textit{Deep Learning}) y aprendizaje 
 automático (\textit{Machine Learning}), que ofrecen una mayor capacidad
  para modelar relaciones no lineales y para integrar información proveniente de
   diversas fuentes. Modelos como Redes Neuronales Recurrentes (RNN), LSTM 
   (\textit{Long Short-Term Memory}) y \textit{Random Forest Regressor} se han 
   aplicado con éxito en la predicción de ventas en entornos de comercio 
   electrónico, debido a su habilidad para capturar dependencias temporales y 
   correlaciones entre variables \citep{hochreiter1997lstm, goodfellow2016deep, breiman2001random, bandara2020forecast}.

De forma más reciente, la literatura destaca los modelos basados en 
transformadores, como \textit{Temporal Fusion Transformer} (TFT) \citep{lim2021temporal}, Informer \citep{zhou2021informer} y N-BEATS \citep{oreshkin2019nbeats}, que alcanzaron resultados competitivos en competencias 
internacionales de series temporales y en aplicaciones industriales.
 Estos modelos incorporan mecanismos de atención que permiten identificar qué 
 variables externas influyen en la demanda y en qué momentos lo hacen, lo que 
 resulta especialmente relevante en contextos con campañas publicitarias, 
 promociones o cambios abruptos en la visibilidad de los productos.

El interés por evaluar rigurosamente estas técnicas ha impulsado la creación de 
competencias a gran escala, como M4 y M5 \citep{makridakis2020m5}, organizadas por la comunidad académica
 y la industria. Estas competencias demostraron que la combinación de enfoques 
 tradicionales con métodos basados en aprendizaje profundo permite alcanzar 
 mejoras significativas en la precisión de los pronósticos, especialmente 
 cuando los datos provienen de entornos comerciales reales.

 Asimismo, el uso de plataformas integradas de datos, tales como Shopify 
 \citep{shopify} y Triple Whale \citep{triplewhale}, permite el desarrollo de 
 soluciones personalizadas que combinan información transaccional, métricas de 
 marketing y comportamiento del usuario. La integración de estas fuentes 
 facilita la incorporación de variables exógenas, como inversión publicitaria, 
 características de los productos, descuentos, actividad de suscripciones o 
 lanzamientos de nuevas variantes. Estas variables contribuyen a mejorar la 
 capacidad predictiva de los modelos.

Entre las soluciones comerciales más reconocidas se destacan Amazon Forecast 
\citep{amazonforecast}, un servicio basado en redes neuronales desarrollado por 
Amazon Web Services \citep{aws} que automatiza la creación de modelos de 
predicción de demanda, y Prophet \citep{taylor2018forecasting}, una herramienta 
de código abierto creada por Meta \citep{meta} que utiliza un enfoque aditivo 
para modelar tendencias y estacionalidades de manera flexible. Ambas 
herramientas representan referencias relevantes en el campo del pronóstico de 
series temporales con datos de negocios.

La tendencia actual apunta hacia el desarrollo de modelos híbridos que combinan 
la capacidad de las redes neuronales para capturar relaciones complejas con la 
interpretabilidad y la robustez de los métodos tradicionales. Esta combinación 
permite enfrentar entornos caracterizados por factores externos altamente 
variables, como campañas digitales, promociones estacionales, descuentos 
temporales o lanzamientos de productos.

En este marco, el trabajo desarrollado se apoya en los enfoques actuales de 
predicción de demanda mediante aprendizaje automático, datos históricos de 
ventas, inversión publicitaria y comportamiento del usuario. El objetivo 
consiste en construir un modelo que represente de forma adecuada las 
tendencias reales del negocio y que funcione como herramienta de apoyo 
para la toma de decisiones estratégicas en la empresa Latech.




\section{Objetivo y alcance}

En esta sección se mencionan los propósitos principales del trabajo y los límites de su implementación.

\subsection{Objetivo del trabajo}

El propósito de este trabajo es desarrollar un modelo de inteligencia artificial que permita predecir con precisión las ventas del producto alimenticio de la empresa Latech, a partir de datos históricos de ventas y de inversión publicitaria. 

Los objetivos específicos incluyen:

\begin{itemize}
\item Construir un sistema predictivo que anticipe la demanda para horizontes de pronóstico corto (1 a 6 meses), período crítico para la planificación operativa de la empresa.
\item Identificar y cuantificar el impacto de las campañas publicitarias en las ventas mediante mecanismos de atención interpretable.
\item Proporcionar a la empresa herramientas para optimizar la planificación de inventario y producción.
\item Reducir los riesgos de faltantes y excesos de stock mediante predicciones confiables.
\item Establecer una base técnica para la toma de decisiones estratégicas basada en datos.
\end{itemize}

\subsection{Alcance del trabajo}

A continuación, se enuncian los items que incluye el trabajo:

\begin{itemize}
    \item Relevamiento y análisis de los datos históricos de ventas de la tienda en línea obtenidos mediante la API de Shopify.
    
    \item Relevamiento y análisis de los datos históricos de inversión publicitaria provenientes de la
API de Triple Whale.

    \item Limpieza, transformación y consolidación de datos provenientes de ambas plataformas.
    
     \item Análisis exploratorio de datos (EDA) para identificar patrones, tendencias y relaciones
entre las variables.
     
     \item Desarrollo y entrenamiento de modelos de inteligencia artificial orientados a la predicción
de ventas.
     
     \item Evaluación comparativa de distintos modelos para seleccionar el que ofrezca la mejor
precisión y capacidad predictiva.
     
     \item Implementación de un módulo funcional integrado al sistema Inventory Tracker de la
empresa, que permita realizar predicciones de ventas para períodos futuros.

     
     \item Documentación técnica del proceso, del modelo elegido y de su uso.
     
     \item Entrega de reportes que expliquen los hallazgos y recomendaciones derivadas del análisis.
     
\end{itemize}

Por otra parte, quedan excluidas del alcance las siguientes actividades:

\begin{itemize}
    \item Implementaciones en tiempo real del modelo, es decir, sistemas que actualicen las
predicciones de forma instantánea ante cada nuevo dato recibido. Sí se contempla la
posibilidad de programar ejecuciones periódicas del modelo (por ejemplo, una vez al día)
para actualizar las predicciones de manera regular.
	\item Garantía de precisión absoluta en las predicciones, ya que la precisión depende de la
calidad y estabilidad de los datos futuros y de factores externos no controlables.
	\item Acciones o recomendaciones específicas sobre estrategias de marketing más allá de lo
inferido de los análisis de datos.
	\item Re-ingeniería de procesos operativos internos de producción o logística.
	\item Optimización de procesos internos de producción o logística, salvo en lo que respecta a la estimación de demanda.

\end{itemize}

